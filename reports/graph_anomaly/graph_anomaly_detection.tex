\documentclass[9pt,twocolumn,twoside]{idsi}

%%%%% THIS FILE WAS AUTOGENERATED %%%%%
\author[1,2]{Bhuvan Venkatesh}
\author[2,3]{Professor Brunner}
\affil[1]{National Center For Supercomputing Applications (NCSA)}
\affil[2]{Laboratory for Computation, Data, and Machine Learning}
\affil[3]{Univeristy of Illinois}
\title{LaTeX yeah For an iDSI Technical Report}

\newcommand{\makecoverpage}{
	\begin{titlepage}
	\center 
	\textsc{\LARGE Illinois Data Science Initiative} \\
	[1.5cm]\textsc{\Large LaTeX yeah For an iDSI Technical Report} \HRule \\
	[0.4cm]{\huge \emph{Version: } 0.0.1 }\\
	\HRule \\
	[1.5cm]\Large \emph{Author(s): } Bhuvan Venkatesh, Professor Brunner \n \\
	[3.0cm] {\large \today} % Date
	%\includegraphics{Logo}\\[1cm] % uncomment if you want to place a logo
	\vfill
	\end{titlepage}
}


\begin{abstract}

\end{abstract}

\begin{document}

\makecoverpage

\maketitle

\section{Introduction}
This template is designed to assist with creating a two-column research article or letter to submit to \emph{iDSI}.

If you have a question while using this template on write\LaTeX{}, please use the help menu (``?'') on the top bar to search for help or ask us a question using the option in the lower right of the editor.

\section{Examples of Article Components}
\label{sec:examples}

The sections below show examples of different article components.

\section{Figures and Tables}

It is not necessary to place figures and tables at the back of the manuscript. Figures and tables should be sized as they are to appear in the final article. Do not include a separate list of figure captions and table titles.

Figures and Tables should be labelled and referenced in the standard way using the \verb|\label{}| and \verb|\ref{}| commands.

\subsection{Sample Figure}

Figure \ref{fig:false-color} shows an example figure.

\begin{figure}[htbp]
\centering
\caption{False-color image, where each pixel is assigned to one of seven reference spectra.}
\label{fig:false-color}
\end{figure}

\subsection{Sample Table}

Table \ref{tab:shape-functions} shows an example table. 

\begin{table}[htbp]
\centering
\caption{\bf Shape Functions for Quadratic Line Elements}
\begin{tabular}{ccc}
\hline
local node & $\{N\}_m$ & $\{\Phi_i\}_m$ $(i=x,y,z)$ \\
\hline
$m = 1$ & $L_1(2L_1-1)$ & $\Phi_{i1}$ \\
$m = 2$ & $L_2(2L_2-1)$ & $\Phi_{i2}$ \\
$m = 3$ & $L_3=4L_1L_2$ & $\Phi_{i3}$ \\
\hline
\end{tabular}
  \label{tab:shape-functions}
\end{table}

\section{Sample Equation}

Let $X_1, X_2, \ldots, X_n$ be a sequence of independent and identically distributed random variables with $\text{E}[X_i] = \mu$ and $\text{Var}[X_i] = \sigma^2 < \infty$, and let
\begin{equation}
S_n = \frac{X_1 + X_2 + \cdots + X_n}{n}
      = \frac{1}{n}\sum_{i}^{n} X_i
\label{eq:refname1}
\end{equation}
denote their mean. Then as $n$ approaches infinity, the random variables $\sqrt{n}(S_n - \mu)$ converge in distribution to a normal $\mathcal{N}(0, \sigma^2)$.

\section{Sample Algorithm}

Algorithms can be included using the commands as shown in algorithm \ref{alg:euclid}.

\begin{algorithm}
\caption{Euclid’s algorithm}\label{alg:euclid}
\begin{algorithmic}[1]
\Procedure{Euclid}{$a,b$}\Comment{The g.c.d. of a and b}
\State $r\gets a\bmod b$
\While{$r\not=0$}\Comment{We have the answer if r is 0}
\State $a\gets b$
\State $b\gets r$
\State $r\gets a\bmod b$
\EndWhile\label{euclidendwhile}
\State \textbf{return} $b$\Comment{The gcd is b}
\EndProcedure
\end{algorithmic}
\end{algorithm}

\section*{Funding Information}
National Science Foundation (NSF) (1263236, 0968895, 1102301); The 863 Program (2013AA014402).

\section*{Acknowledgments}

Formal funding declarations should not be included in the acknowledgments but in a Funding Information section as shown above. The acknowledgments may contain information that is not related to funding:

The authors thank H. Haase, C. Wiede, and J. Gabler for technical support.

\section*{Supplemental Documents}
\emph{Optica} authors may include supplemental documents with the primary manuscript. For details, see \href{http://www.opticsinfobase.org/submit/style/supplementary-materials-optica.cfm}{Supplementary Materials in Optica}. To reference the supplementary document, the statement ``See Supplement 1 for supporting content.'' should appear at the bottom of the manuscript (above the references).

\section*{References}

For references, you may add citations manually or use BibTeX. E.g. \cite{Zhang:14}.

Note that letter submissions to \emph{Optica} use an abbreviated reference style. Citations to journal articles should omit the article title and final page number; this abbreviated reference style is produced automatically when the \texttt{$\setminus$setboolean\{shortarticle\}\{true\}} option is selected in the template, if you are using a .bib file for your references. 

However, full references (to aid the editor and reviewers) must be included as well on an informational page that will not count against page length; again this will be produced automatically if you are using a .bib file and have the \texttt{$\setminus$setboolean\{shortarticle\}\{true\}} option selected.


\end{document}